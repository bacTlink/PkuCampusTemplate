\documentclass[a4paper,12pt]{article}

\usepackage{graphicx}
\usepackage{hyperref}
\usepackage{caption,subcaption}
\usepackage{xcolor}
\usepackage{ctex}
\usepackage{geometry}
\usepackage{tikz}
\usepackage{comment}
\usepackage{xifthen}
\usepackage{titlesec}
\usepackage{forloop}
\usepackage{lastpage}
\usepackage{totcount}
\usepackage[inline, shortlabels]{enumitem}
\usepackage{amsmath}
\usepackage{array}

% section counter
\newtotcounter{section_cnt}

% set page
\geometry{left=2cm,right=2cm,top=3cm,bottom=2cm}

% section format
\renewcommand\thesection{\Alph{section}}
\titleformat{\section}[block]{\Huge\filcenter\textbf}{\thesection.}{1em}{}
\titleformat{\subsection}[hang]{\bfseries}{}{1em}{}

% count section
\usepackage{letltxmacro}
\LetLtxMacro{\oldsection}{\section} \renewcommand{\section}[1]{\oldsection{#1}\label{\thesection}\stepcounter{section_cnt}}

% input tex files
\renewcommand{\include}[1]{\input{"#1.tex"}

}

% set enumerate padding
\setenumerate[1]{itemsep=0pt,partopsep=0pt,parsep=\parskip,topsep=0pt}
\setitemize[1]{itemsep=0pt,partopsep=0pt,parsep=\parskip,topsep=0pt}

\usepackage{xargs}

\newcommandx{\MakeLimits}[2][1=1000 ms, 2=65536 kB]{
    \begin{center}
        \large{Time Limit: #1 \hspace{20pt} Memory Limit: #2}
    \end{center}
}

\title{\textbf{PKU Campus 2019\\Contest Problem Set}}
\date{\textbf{May 12th, 2019}}
\author{\textbf{Peking University}}

\begin{document}
\maketitle

\label{FirstPage}

\vspace{90pt}
\begin{center}
This problem set contains \number\totvalue{section_cnt} problems; pages are numbered from \pageref{FirstPage} to \pageref{LastPage}.
\end{center}

\vspace{90pt}
\textbf{Attention:}
\begin{enumerate}
    \item The input must be read from standard input.
    \item The output must be written to standard output.
    \item If you use long long in your gcc/g++ program, make sure you use ``\%lld'' instead of ``\%I64d'' while reading or writing long long value.
    \item Due to possible changes during the contest, please refresh the web frequently.
\end{enumerate}

\clearpage{}

\newgeometry{left=2cm,right=2cm,top=4.8cm,bottom=2cm}

\newcommand{\addline}[1]{\rule{0pt}{2.1em} \Large{\ref{#1}} & 
\Large{\nameref{#1}} & \Large{\pageref{#1}} \\ [1ex] }

\begin{table}[ht]

\centering
\caption*{\Huge{Problem List}}

\newcounter{it}
\begin{tabular}{>{\centering\arraybackslash}p{1.8cm} c >{\centering\arraybackslash}p{1.8cm}}
\hline
\rule{0pt}{2.4em} \LARGE{ID} & \LARGE{Title} & \LARGE{Page} \\ [1ex]
\hline
  \setcounter{it}{1}
  \whiledo{\theit<\totvalue{section_cnt}}{
    \addline{\Alph{it}}
    \stepcounter{it}
  }
  \addline{\Alph{it}}
  \hline
\end{tabular}

\end{table}
\restoregeometry
\clearpage{}

\include{ProblemList}

\end{document}

